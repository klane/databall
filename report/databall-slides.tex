\documentclass{beamer}
\usetheme{Madrid}
% \usecolortheme{spruce}

% Change base colour beamer@blendedblue (originally RGB: 0.2,0.2,0.7)
\colorlet{beamer@blendedblue}{green!40!black}

\usepackage[citestyle=authoryear, sorting=none]{biblatex}
\addbibresource{databall.bib}

\title{DataBall}
\subtitle{Predicting NBA Winners with Data}
\author{Kevin Lane}

\begin{document}

\begin{frame}
\titlepage
\end{frame}

\begin{frame}
\frametitle{Outline}
\tableofcontents
\end{frame}

\section{Introduction}

\subsection{Background}
\begin{frame}
\frametitle{Background}
\begin{itemize}
    \item Sports analytics began in professional baseball, most notably with the work of Bill James\footcite{james}
    \begin{itemize}
        \item James coined the term sabermetrics as ``the search for objective knowledge about baseball"
        \item James selected the name to honor the Society for American Baseball Research (SABR)
    \end{itemize}
    \item Gained widespread adoption after Billy Beane implemented James' ideas and led the Oakland Athletics to a record winning streak\footcite{lewis}
    \item Analytics has since spread to other sports and its impact is evidenced by several examples:
    \begin{itemize}
        \item MLB's increased attention to on-base percentage beginning in the Moneyball era of the early 2000s
        \item The rise of the three-point shot and subsequent fall of the midrange jumper in the NBA
        \item Increased use of short, high percentage passes in the NFL
    \end{itemize}
\end{itemize}
\end{frame}

\begin{frame}
\frametitle{Background}
\begin{block}{What makes sports an attractive testbed for machine learning?}
According to Nate Silver, ``sports nerds have it easy."\footnotemark
\begin{enumerate}
    \item ``Sports has awesome data."
    \item ``In sports, we know the rules."
    \item ``Sports offers fast feedback and clear marks of success."
\end{enumerate}
\end{block}
\footcitetext{silver-online}
\vspace{-0.5cm}
\begin{block}{Why the NBA?}
\begin{itemize}
    \item Easily the most deterministic of the major American sports
    \item The NBA provides a wealth of advanced stats and player tracking data on their website
    \item The season is long enough at 82 games that sample size is not as much of a concern as in the NFL, who claim to have parity, but also only play 16 regular season games
\end{itemize}
\end{block}
\end{frame}

\subsection{Process}
\begin{frame}
\frametitle{Process}
\begin{itemize}
    \item
\end{itemize}
\end{frame}

\section{Data}

\subsection{Data Wrangling}
\begin{frame}
\frametitle{Data Wrangling}
\begin{itemize}
    \item
\end{itemize}
\end{frame}

\subsection{Data Exploration}
\begin{frame}
\frametitle{Data Exploration}
\begin{columns}
\column{0.5\textwidth}
\begin{itemize}
    \item The top figure shows that the home team winning percentage is remarkably consistent
    \begin{itemize}
        \item Simply predicting the home team wins will yield about 60\% accuracy
        \item This provides a good baseline; any predictive model worth implementing should beat this
    \end{itemize}
    \item The bottom figure shows that team performance (according to SRS) very closely resembles a normal distribution
    \begin{itemize}
        \item An SRS of zero indicates an average team
    \end{itemize}
\end{itemize}
\column{0.5\textwidth}
\vspace{-0.5cm}
\begin{figure}
\includegraphics[width=60mm]{../docs/assets/images/data-exploration/home-win-pct.png}
\end{figure}
\vspace{-0.5cm}
\begin{figure}
\includegraphics[width=60mm]{../docs/assets/images/data-exploration/srs-distribution.png}
\end{figure}
\end{columns}
\end{frame}

\begin{frame}
\frametitle{Data Exploration}
\begin{itemize}
    \item The plots below show kernel density estimations (KDE) of SRS split between home team wins and losses
    \item The dark region to the bottom right of the origin for home team wins shows above-average home teams tend to beat below-average visitors
    \item The opposite appears in the KDE of home team losses
\end{itemize}
\begin{figure}
\includegraphics[scale=0.35]{../docs/assets/images/data-exploration/srs-win-loss-kde.png}
\end{figure}
\end{frame}

\section{Model Selection}

\subsection{Feature Selection}
\begin{frame}
\frametitle{Feature Selection}
\begin{itemize}
    \item
\end{itemize}
\end{frame}

\subsection{Parameter Tuning}
\begin{frame}
\frametitle{Parameter Tuning}
\begin{itemize}
    \item
\end{itemize}
\end{frame}

\section{Results}

\subsection{Model Performance}
\begin{frame}
\frametitle{Model Performance}
\begin{itemize}
    \item
\end{itemize}
\end{frame}

\subsection{Comparison to Published Results}
\begin{frame}
\frametitle{Comparison to Published Results}
\begin{itemize}
    \item
\end{itemize}
\end{frame}

\section{References}

\begin{frame}[t]
\frametitle{References}
\printbibliography
\end{frame}

\end{document}
